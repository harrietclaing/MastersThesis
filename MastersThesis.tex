\documentclass[11pt,preprint, authoryear]{elsarticle}

\usepackage{lmodern}
%%%% My spacing
\usepackage{setspace}
\setstretch{1.2}
\DeclareMathSizes{12}{14}{10}{10}

% Wrap around which gives all figures included the [H] command, or places it "here". This can be tedious to code in Rmarkdown.
\usepackage{float}
\let\origfigure\figure
\let\endorigfigure\endfigure
\renewenvironment{figure}[1][2] {
    \expandafter\origfigure\expandafter[H]
} {
    \endorigfigure
}

\let\origtable\table
\let\endorigtable\endtable
\renewenvironment{table}[1][2] {
    \expandafter\origtable\expandafter[H]
} {
    \endorigtable
}


\usepackage{ifxetex,ifluatex}
\usepackage{fixltx2e} % provides \textsubscript
\ifnum 0\ifxetex 1\fi\ifluatex 1\fi=0 % if pdftex
  \usepackage[T1]{fontenc}
  \usepackage[utf8]{inputenc}
\else % if luatex or xelatex
  \ifxetex
    \usepackage{mathspec}
    \usepackage{xltxtra,xunicode}
  \else
    \usepackage{fontspec}
  \fi
  \defaultfontfeatures{Mapping=tex-text,Scale=MatchLowercase}
  \newcommand{\euro}{€}
\fi

\usepackage{amssymb, amsmath, amsthm, amsfonts}

\def\bibsection{\section*{References}} %%% Make "References" appear before bibliography


\usepackage[round]{natbib}

\usepackage{longtable}
\usepackage[margin=2.3cm,bottom=2cm,top=2.5cm, includefoot]{geometry}
\usepackage{fancyhdr}
\usepackage[bottom, hang, flushmargin]{footmisc}
\usepackage{graphicx}
\numberwithin{equation}{section}
\numberwithin{figure}{section}
\numberwithin{table}{section}
\setlength{\parindent}{0cm}
\setlength{\parskip}{1.3ex plus 0.5ex minus 0.3ex}
\usepackage{textcomp}
\renewcommand{\headrulewidth}{0.2pt}
\renewcommand{\footrulewidth}{0.3pt}

\usepackage{array}
\newcolumntype{x}[1]{>{\centering\arraybackslash\hspace{0pt}}p{#1}}

%%%%  Remove the "preprint submitted to" part. Don't worry about this either, it just looks better without it:
\makeatletter
\def\ps@pprintTitle{%
  \let\@oddhead\@empty
  \let\@evenhead\@empty
  \let\@oddfoot\@empty
  \let\@evenfoot\@oddfoot
}
\makeatother

 \def\tightlist{} % This allows for subbullets!

\usepackage{hyperref}
\hypersetup{breaklinks=true,
            bookmarks=true,
            colorlinks=true,
            citecolor=blue,
            urlcolor=blue,
            linkcolor=blue,
            pdfborder={0 0 0}}


% The following packages allow huxtable to work:
\usepackage{siunitx}
\usepackage{multirow}
\usepackage{hhline}
\usepackage{calc}
\usepackage{tabularx}
\usepackage{booktabs}
\usepackage{caption}


\newenvironment{columns}[1][]{}{}

\newenvironment{column}[1]{\begin{minipage}{#1}\ignorespaces}{%
\end{minipage}
\ifhmode\unskip\fi
\aftergroup\useignorespacesandallpars}

\def\useignorespacesandallpars#1\ignorespaces\fi{%
#1\fi\ignorespacesandallpars}

\makeatletter
\def\ignorespacesandallpars{%
  \@ifnextchar\par
    {\expandafter\ignorespacesandallpars\@gobble}%
    {}%
}
\makeatother

\newlength{\cslhangindent}
\setlength{\cslhangindent}{1.5em}
\newenvironment{CSLReferences}%
  {\setlength{\parindent}{0pt}%
  \everypar{\setlength{\hangindent}{\cslhangindent}}\ignorespaces}%
  {\par}


\urlstyle{same}  % don't use monospace font for urls
\setlength{\parindent}{0pt}
\setlength{\parskip}{6pt plus 2pt minus 1pt}
\setlength{\emergencystretch}{3em}  % prevent overfull lines
\setcounter{secnumdepth}{5}

%%% Use protect on footnotes to avoid problems with footnotes in titles
\let\rmarkdownfootnote\footnote%
\def\footnote{\protect\rmarkdownfootnote}
\IfFileExists{upquote.sty}{\usepackage{upquote}}{}

%%% Include extra packages specified by user

%%% Hard setting column skips for reports - this ensures greater consistency and control over the length settings in the document.
%% page layout
%% paragraphs
\setlength{\baselineskip}{12pt plus 0pt minus 0pt}
\setlength{\parskip}{12pt plus 0pt minus 0pt}
\setlength{\parindent}{0pt plus 0pt minus 0pt}
%% floats
\setlength{\floatsep}{12pt plus 0 pt minus 0pt}
\setlength{\textfloatsep}{20pt plus 0pt minus 0pt}
\setlength{\intextsep}{14pt plus 0pt minus 0pt}
\setlength{\dbltextfloatsep}{20pt plus 0pt minus 0pt}
\setlength{\dblfloatsep}{14pt plus 0pt minus 0pt}
%% maths
\setlength{\abovedisplayskip}{12pt plus 0pt minus 0pt}
\setlength{\belowdisplayskip}{12pt plus 0pt minus 0pt}
%% lists
\setlength{\topsep}{10pt plus 0pt minus 0pt}
\setlength{\partopsep}{3pt plus 0pt minus 0pt}
\setlength{\itemsep}{5pt plus 0pt minus 0pt}
\setlength{\labelsep}{8mm plus 0mm minus 0mm}
\setlength{\parsep}{\the\parskip}
\setlength{\listparindent}{\the\parindent}
%% verbatim
\setlength{\fboxsep}{5pt plus 0pt minus 0pt}



\begin{document}



\begin{frontmatter}  %

\title{Helping You Write Academic Papers in R using Texevier}

% Set to FALSE if wanting to remove title (for submission)




\author[Add1]{Nico Katzke\footnote{\textbf{Contributions:}
  \newline \emph{The authors would like to thank no institution for
  money donated to this project. Thank you sincerely.}}}
\ead{nfkatzke@gmail.com}

\author[Add1,Add2]{John Smith}
\ead{John@gmail.com}

\author[Add1,Add2]{John Doe}
\ead{Joe@gmail.com}



\address[Add1]{Prescient Securities, Cape Town, South Africa}
\address[Add2]{Some other Institution, Cape Town, South Africa}

\cortext[cor]{Corresponding author: Nico Katzke\footnote{\textbf{Contributions:}
  \newline \emph{The authors would like to thank no institution for
  money donated to this project. Thank you sincerely.}}}

\begin{abstract}
\small{
Abstract to be written here. The abstract should not be too long and
should provide the reader with a good understanding what you are writing
about. Academic papers are not like novels where you keep the reader in
suspense. To be effective in getting others to read your paper, be as
open and concise about your findings here as possible. Ideally, upon
reading your abstract, the reader should feel he / she must read your
paper in entirety.
}
\end{abstract}

\vspace{1cm}


\begin{keyword}
\footnotesize{
Multivariate GARCH \sep Kalman Filter \sep Copula \\
\vspace{0.3cm}
}
\footnotesize{
\textit{JEL classification} L250 \sep L100
}
\end{keyword}



\vspace{0.5cm}

\end{frontmatter}



%________________________
% Header and Footers
%%%%%%%%%%%%%%%%%%%%%%%%%%%%%%%%%
\pagestyle{fancy}
\chead{}
\rhead{}
\lfoot{}
\rfoot{\footnotesize Page \thepage}
\lhead{}
%\rfoot{\footnotesize Page \thepage } % "e.g. Page 2"
\cfoot{}

%\setlength\headheight{30pt}
%%%%%%%%%%%%%%%%%%%%%%%%%%%%%%%%%
%________________________

\headsep 35pt % So that header does not go over title




\hypertarget{introduction}{%
\section{\texorpdfstring{Introduction
\label{Introduction}}{Introduction }}\label{introduction}}

(approx. 800)

The primary objective of this research assignment is to model how
interactions in the political economy inhibit the advancement of
environmental policies. Ultimately, we wish to understand how the
frictions and dominance of certain countries in the global area impacts
global environmental policy. For example, the US is a relatively smaller
country with a large dominant role in the global arena that has a
greater voting weight, compared to larger countries geographically with
less dominance politically. This results in sub-optimal outcomes such as
dominant countries contributing to pollution in larger countries that
are geographically positioned in relatively more vulnerable positions.
The model we set up first considers the simplest case of a two-period
optimisation problem. There is a world with abundant resources, where
consumers believe that they can consume a lot today and tomorrow, and
information about the state of the world is aggregated perfectly. As we
continue to model this problem, frictions are added. For example, the
state of the world may not be aggregated correctly to the voters. For
our model, the focus is how the political system works when there is the
issue of environmental degradation. Accordingly, the literature on both
sides of the sustainability of the political economy is considered.
Namely, the literature on exhaustible resources (including the optimal
depletion thereof and the cake-eating optimisation problem) and the
political economy (including the aggregation of political information
and strategic voting).

\hypertarget{literature-review}{%
\section{Literature review}\label{literature-review}}

(approx. 2000)

Pindyck (1978) models two optimal paths for using non-renewable
resources depending on the initial resource endowment and the different
marginal costs of extraction. If the initial reserve is large, then this
will cause extraction costs to be low and thereby price will slowly rise
over time. On the other hand, if the initial reserve has been depleted
and is small, both the price and the extraction costs will be high. As
reserves decline, the production cost for using resources increases and
potential profits reduces. Accordingly, as marginal costs increase for
extraction, producers will begin to conserve non-renewable resources
because the marginal benefit thereof is smaller. Therefore, for
non-renewable resources with a fixed reserve its time horizon can be
increased with economic incentives such as profit opportunities that
increase exploratory activity; ``potential reserves'' are unlimited.
However, the model does not include the situation in which the
non-renewable quantities are uncertain.

Kemp (1976) was the first to consider the cake-eating optimisation
problem for an exhaustible resource of an unknown size and the idea has
since been developed in the literature. Loury (1978) investigates the
problem of optimal planning when faced with two issues in this regard;
(i) the possibility of exhaustion of the resource, and (ii) learning
about the distribution of reserves over time through exploration and
extraction activities. Under certainty when a resource's reserve is
known, the possibility of exhaustion occurs only asymptotically.
However, when the resource base is of unknown size, the date on which it
will have been completely exploited is also a random variable. The
choice of the rate of consumption is necessarily affected by the impact
of the consumption rate on the probability of `crisis', or likelihood of
reaching the tipping point. Therefore, this paper demonstrates that the
possibility of premature exhaustion alters the requirements of an
optimal consumption path. The model used in the paper includes the date
upon which the resource is exhausted as a choice variable and specifies
some minimal level of resource consumption that is necessary to sustain
an economy. ``Set the objective function as the maximisation of the
expected value of the integral of discounted utility. Employ dynamic
programming techniques to deduce some comparative static results'' The
uncertain reserve is a random variable that is drawn from a cumulative
density function.

Kumar (2005) reconsiders Kemp's use of an infinite planning horizon by
instead using a hazard function which is defined as the probability of
reaching the climate tipping point. The paper argues that `cake-eating
under uncertainty' has two main aspects: (i) optimal planning horizon
and (ii) the characterisation of optimal program. The former aspect of
the (i) optimal planning horizon can be assumed to extend far enough
into the future to allow for the resource to be fully exhausted at some
point. By employing a hazard function to model the problem, this
planning horizon over which consumption of the exhaustible resource is
positive has been found to be either finite or infinite. The findings of
Kumar (2005) are that if an uncertain resource stock is finite and the
optimal planning horizon is finite, hazard rates increase in unbounded
fashion, and marginal utility of extraction and consumption at zero is
finite. However, if an uncertain resource stock size is unbounded, the
planning horizon is infinite. Thus, the optimal rate of extraction and
consumption over time generally moves monotonically in the opposite
direction to the hazard rate.

Adler \& Treich (2017) consider additional elements that are not
included in Kumar (2005) which are involved in the decision for optimal
climate policy. There are three dimensions of climate policy modelled:
(i) equity, (ii) time, and (iii) risk which each impact how risky
consumption is allocated intertemporally. The cake-eating problem for an
uncertain quantity of resources with a social planner allocating between
the current and future generation is modelled in terms of three
different social welfare functions. The first generation is given a
determinate amount of consumption, but the second generation is given
the risky remainder. The methodology used is a prioritarian social
welfare function (SWF) which is the sum of some increasing, concave
function of utility for the well-being of a generation. To find the
optimal consumption allocation, Adler \& Treich (2017) compare statics
analysis for the different functions.

Lemoine \& Traeger (2014) address the need to integrate policy with
possibility of climate tipping points into a benchmark integrated
assessment model'' to analyse the intertemporal trade-offs that
characterise climate policy decisions. Crossing the threshold shifts the
world permanently to a new altered system with different dynamics to
pre-threshold. The paper ``analytically demonstrated that tipping points
affect optimal policy via two channels: the differential welfare impact
(DWI) recognizes that today's policy choices affect welfare in the case
that a tipping point occurs, and the marginal hazard effect (MHE)
recognizes that today's policy choices affect the probability of
crossing the threshold.'' endogenises welfare and probability of tipping
point.

On the political economy side of the literature, Piketty (1999) provides
a short survey of recent contributions to the literature related to
political institutions as an information-aggregation mechanism.
Condorcet (1785) was the first to posit that political institutions have
a constructive role in efficiently aggregating information in a society.
His main contribution to the literature was termed the `Condorcet Jury
Theorem' which states that under free elections, the probability that
the the policy that is preferred by the majority will win by majority
vote tends to one as the number of individuals in the population tends
to infinity. This theorem posits democracy is an efficient
information-aggregation system if it is assumed that individuals are
homogenous in their initial prior beliefs about the state of the world,
each receives a signal drawn from the same conditional distribution and
that it is realistic that more than half of citizens are right/ more
than half of the time.

Many models of political economy include different stages to analyse how
voting impacts policies but vary according to how voters' derive utility
(Lohmann, 1994; Besley \& Coate, 1998; Razin, 2003; Buchholz et al,
2005). Lohmann (1994) investigated the impact of the information
aggregated by political action before a vote on whether votes cast are
more or less accurate, where accuracy is defined in terms of reflecting
voters' preferences. The paper models that each individual has a loss
function which dictates the loss if policy outcome is not what the
individual desired and the private cost if the individual engaged in
political action. The equilibrium point is where an individual's
political action strategy minimises her expected loss after the
political action stage, and thus an individual only engages in political
action if the cost of doing so is outweighed by the potential benefit of
achieving the political outcome she wants. Razin (2003) models elections
with two potential candidates in a one-dimensional policy space. Voting
behaviour is strategic as voters are motivated by both the election and
signalling implications of voting. Razin (2003) assumes that voters are
privately and imperfectly informed about a common shock which impacts
voters' preferences. Voters' private signals are drawn independently
from a distribution and are conditional on the common shock. Razin
(2003) also uses the policy gap idea (how far is the actual policy
outcome from the voter's preference/ideal) in a utility function.

To analyse voter behaviour, Lohmann (1994) uses game theoretical best
responses at each stage are determined. For example, at the political
action stage, their best response is a function of the `type' of voter
they are and the individual preferences they have. Razin (2003) also
uses game theory with two stage process: First stage = common shock
dictated by nature and voters receive and observe independent and
private signals that are conditional upon this shock. Second stage =
voters cast ballots simultaneously for either candidate L or R (cannot
abstain from voting). Candidate with majority wins and chooses a policy
according to votes (assuming candidates are responsive).

Buchholz et al (2005) investigates the policy outcome that may arise
from international environmental agreements when governments are
democratically elected by citizens. This paper considers two pertinent
themes, namely the case in which a voter may be incentivised to support
a political candidate who is less environmentally inclined than the
voter's own preferences and the case in which the elected candidate pays
no attention to environmental policies and thus international agreements
are rendered ineffective. The paper argues that efficient allocation of
resources in a sustainable way requires co-operation between countries
but that there is a far way to go before international environmental
agreements are fully enforced and functional. A possible reason for
ineffective IEAs is that voters are incentivised to deliberately support
candidates with different environmental preferences to their own in
order to improve their country's bargaining position in international
negotiations.

Methodology is to use median-voter approach (``If all voters have
single-peaked preferences then the alternative that is the most
preferred by the median voter will defeat any other alternative in a
pairwise majority vote. See also voting.'') with Nash bargaining in
simple two country model to understand the effects of democratically
elected governments on IEAs' effectiveness. In first stage, citizens
from both countries elect politicians and then in second stage, the
elected politicians negotiate over reducing pollution/environmental
policies. In the third stage, if agreement reached then IEA becomes
binding but otherwise, countries independently choose policies and
accordingly defines a threat point for bargaining.

The main conclusion is that each voter supports a candidate that is less
environmentally conscious than they are because conducting strategic
voting improves their country's bargaining position, which is due to the
free-rider problem that simply moves from the politicians to the voters.
The paper compares the isolationist case in which there is no IEA
between countries to the bargaining or co-operative case with an IEA and
finds that countries will be incentivised to improve their bargaining
position in an IEA by introducing less green policies, than they would
otherwise in a unilateral arrangement.

The paper models the electoral system in a stylised way which means that
the winning candidate is determined when they win every pairwise
comparison with all other candidates and assumed that the competing
candidates have different types ranging between 0 and 1. On the other
hand, Besley \& Coate (1998) set up a model of representative democracy
in which citizens are all able to avail themselves as candidates to run
for public office. The model assumes that candidates must credibly
commit to implementing their preferred policy if they win the election
and voters vote accordingly. Voters derive utility from the ultimate
policy outcome and the winning candidate's identity, and derive
disutility from the cost of running for office if they decide to do so.
There is strategic voting because voters' voting decisions are optimal
if they are a best response, given the rest of voters' decisions. The
paper defines the equilibrium policy choice as efficient according to a
Pareto optimal definition; if there are no alternative policy choices in
the present period that could increase the expected utility of all
citizens conditional on future policies that are democratic in nature.
However, this model assumes complete information. It recommends that
future research incorporates uncertainty regarding voters' preferences
and endogenises the formation of parties or candidates that run for
office.

\hypertarget{model-set-up-of-sustainable-consumption-in-global-economy}{%
\section{Model set-up of sustainable consumption in global
economy}\label{model-set-up-of-sustainable-consumption-in-global-economy}}

The state of the world will first be modelled with infinite resources
and perfect information. Thereafter, frictions will be added to make
resources exhaustible, and information is aggregated imperfectly. The
second specification of the model will add the political economy and
voting decisions, as well as politicians who exhibit self-interest and
non-responsiveness to information about voter's preferences about
climate policy.

Two levels of uncertainty at this point: (i) what is the true threshold?
Determined by nature. (ii) what opinion do most voters hold? Ie are
majority pessimistic about the costs associated with consuming
unsustainably? Assuming voter sincerity (ie voting according to
beliefs/signals)

\hypertarget{economy}{%
\subsection{Economy}\label{economy}}

In order to model sustainability in the political economy, we first set
up a simple model of sustainable consumption over two periods
\(t = 1, 2\). The choice variable in this economy is consumption in
period one (\(c_1\)) and consumption in period two (\(c_2\)) is
stochastic and dependent on the value of \(c_1\) as follows. \(c_1\) is
defined as sustainable if it is below an unknown threshold value
(\(\tilde{c}\)). This is because if \(c_1\) is sustainable, then \(c_2\)
can also be set at the same level (\(c_1\) = \(c_2\)). However, if
\(c_1\) is unsustainable, and exceeds \(\tilde{c}\) in period one, then
the economy will face some penalty in period two (\(\underline{c}\)),
and thus \(c_2\) will be much lower than \(c_1\)
(\(c_1> \underline{c}\)). Thus, we can model \(c_2\) as a function of
\(c_1\) as follows:

\[
c_2\left(c_1\right)= \begin{cases}c_1 & \text { if } c_1 \leq \tilde{c} \\ \underline{c} & \text { otherwise }\end{cases}
\]

Under perfect information, the optimal choice of c1 should be set at the
true highest threshold (cH = c1) to maximise sustainable consumption in
period one and to ensure sustainable consumption will be possible in
period two. The lowest threshold value cL represents the safe level of
consumption at which the probability of a crisis is very low. The
highest threshold value cH represents the maximum consumption value that
may be consumed in period one without incurring a penalty and thus
reduced consumption in period two. Consumers The decision problem faced
by consumers is to maximise their lifetime utility by their choice of
consumption in period one.

\[
U\left(c_1\right)=u\left(c_1\right)+\mathbb{E} u\left(c_2\left(c_1\right)\right)
\]

where the utility derived from consumption is **function form and p
represents relative risk aversion

\[
U(C)=\frac{c^(1-\rho)-1}{1-\rho}
\] Due to the uncertainty in the problem, because there is uncertainty
about how likely climate crisis is, consumers hold beliefs about the
optimal level of consumption in period one and the probability of crisis
at that level of consumption. These beliefs are determined by private
signals \(\hat{\underline{c}}_k\) that consumers receive which are noisy
and conceal the true value \(\underline{c}\), ie information aggregation
is not perfect/efficient \[
\hat{\underline{c}}_k=\underline{c}+\varepsilon_k
\] with \(\mathbb{E}\left(\varepsilon_k\right)=0\)

This is a realistic assumption because in the real world consumers are
exposed to a variety of news sources (noisy) and different headlines
which predict different outcomes for the climate in the future (*ref).
In this model, the optimal consumption point across individuals is
assumed to be uniformly distributed between the low threshold and high
threshold. The low and high thresholds of consumption in period one are
set as exogenous parameters, and are set to 0.5 and 1.5 respectively.
Accordingly, the probability of reaching the tipping point is 0 if the
choice of c1 is below the lowest possible threshold which means there is
no chance of a climate crisis and 1 if c1 exceeds the highest possible
threshold which means that climate crisis is inevitable.

\[
\operatorname{Pr}\left(\tilde{c} \leq c_1\right)=\left\{\begin{array}{cc}
0 & \text { if } c_1 \leq c_L \\
\frac{c_1-c_L}{c_H-c_L} & \text { if } c_1 \in\left(c_L, c_H\right] \\
1 & \text { otherwise }
\end{array}\right.
\]

From the signals that are drawn randomly for each individual, there are
different associated optimal probability of crisis points. In other
words, the different signals that individuals draw determine their
belief about which probability of crisis is optimal.

\[
U\left(c_1 \mid \hat{\underline{c}}_{k}\right)=u\left(c_1\right)+\operatorname{Pr}\left(\tilde{c} \leq c_1\right) u\left(\hat{\underline{c}}_{k}\right)+\left(1-\operatorname{Pr}\left(\tilde{c} \leq c_1\right)\right) u\left(c_1\right)
\]

For example, Figure 1 shows the cumulative density function of the
probability of crisis and it indicates that individuals who receive
higher signals hold a more optimistic belief about how bad the climate
crisis may be, as they believe that the probability of a climate crisis
occurring can be greater at the optimum point. An individual's optimal
probability of crisis depends on the signal provided to the individual.
The higher the signal, the higher the optimal consumption point and thus
the higher the optimal probability of crisis. This is realistic because
individual's that have been provided with information that indicates
that climate crisis is unlikely will consume more and thus be
comfortable with a higher probability of crisis.

\begin{center}
Figure 1
\end{center}

\includegraphics{Figure1base.jpg}\{width=50\%\}

\begin{center}
Table 2: Model comparison
\end{center}

\begin{longtable}[]{@{}
  >{\raggedright\arraybackslash}p{(\columnwidth - 8\tabcolsep) * \real{0.2150}}
  >{\centering\arraybackslash}p{(\columnwidth - 8\tabcolsep) * \real{0.1495}}
  >{\centering\arraybackslash}p{(\columnwidth - 8\tabcolsep) * \real{0.2150}}
  >{\centering\arraybackslash}p{(\columnwidth - 8\tabcolsep) * \real{0.2150}}
  >{\centering\arraybackslash}p{(\columnwidth - 8\tabcolsep) * \real{0.2056}}@{}}
\toprule()
\begin{minipage}[b]{\linewidth}\raggedright
\end{minipage} & \begin{minipage}[b]{\linewidth}\centering
Global democracy
\end{minipage} & \begin{minipage}[b]{\linewidth}\centering
Two countries 0.1 share
\end{minipage} & \begin{minipage}[b]{\linewidth}\centering
Two countries 0.5 share
\end{minipage} & \begin{minipage}[b]{\linewidth}\centering
Two countries 0.9 bias
\end{minipage} \\
\midrule()
\endhead
Period 1 consumption & 0.9499 & 1.0363 & 1.2134 & 1.2234 \\
Probability of crisis & 0.4499 & 0.5363 & 0.7134 & 0.7234 \\
\bottomrule()
\end{longtable}

To reference a section, you have to set a label using
``\textbackslash label'\,' in R, and then reference it in-text as
e.g.~referencing a later section, Section \ref{Meth}.

Writing in Rmarkdown is surprizingly easy - see
\href{https://www.rstudio.com/wp-content/uploads/2015/03/rmarkdown-reference.pdf}{this
website} cheatsheet for a summary on writing Rmd writing tips.

\hypertarget{methodology}{%
\section{\texorpdfstring{Methodology
\label{Meth}}{Methodology }}\label{methodology}}

\hypertarget{subsection}{%
\subsection{Subsection}\label{subsection}}

Equations should be written as such:

\begin{align}
\beta = \sum_{i = 1}^{\infty}\frac{\alpha^2}{\sigma_{t-1}^2} \label{eq1} \\
\int_{x = 1}^{\infty}x_{i} = 1 \notag
\end{align}

If you would like to see the equations as you type in Rmarkdown, use \$
symbols instead (see this for yourself by adjusted the equation):

\[
\beta = \sum_{i = 1}^{\infty}\frac{\alpha^2}{\sigma_{t-1}^2} \\
\int_{x = 1}^{\infty}x_{i} = 1
\]

\hypertarget{conclusion}{%
\section{Conclusion}\label{conclusion}}

I hope you find this template useful. Remember, stackoverflow is your
friend - use it to find answers to questions. Feel free to write me a
mail if you have any questions regarding the use of this package. To
cite this package, simply type citation(``Texevier'') in Rstudio to get
the citation for Katzke (\protect\hyperlink{ref-Texevier}{2017}) (Note
that uncited references in your bibtex file will not be included in
References).

\newpage

\hypertarget{references}{%
\section*{References}\label{references}}
\addcontentsline{toc}{section}{References}

\hypertarget{refs}{}
\begin{CSLReferences}{1}{0}
\leavevmode\vadjust pre{\hypertarget{ref-Texevier}{}}%
Katzke, N.F. 2017. \emph{{Texevier}: {P}ackage to create elsevier
templates for rmarkdown}. Stellenbosch, South Africa: Bureau for
Economic Research.

\end{CSLReferences}

\bibliography{Tex/ref}





\end{document}
